\documentclass{article}

\usepackage{graphicx}
\usepackage[utf8]{inputenc}
\usepackage[a4paper, top=4cm, bottom=4cm, left=4cm, right=4cm]{geometry}

\title{
    \textbf{\textit{SafeStreet}} \\
    \textbf{RASD document}}

\date{Academic year: 2019 - 2020}
\author{
    Dario Miceli Pranio \\
    Pierriccardo Olivieri
}

\begin{document}
\pagenumbering{gobble}

\maketitle

%%%%%%%%%% LOGO POLIMI %%%%%%%%%%
\begin{figure}[h!]
    \centering
    \includegraphics[scale=0.5]{img/logo.png}
\end{figure}

\newpage
\pagenumbering{arabic}
\tableofcontents

\newpage
%%%%%%%%%% CHAPTER 1 %%%%%%%%%%
\section{Introduction}
This is the RASD document for \textit{SafeStreet}, 
it provides a general view about key aspects of the project. The purpose of this document is to formalize a description
of the system's requirements both functional and non-funcional. In the following pages will be covered goals of the 
application with respect to phenomena. This document is addressed to developers as a guideline to implement the requirements
that follows.  

\subsection{Purpose}
SafeStreets is a service that aims to provide users with the possibility to notify authorities when traffic 
violations occur, and in particular parking violations. The application's goal is achieved by allowing users 
to share photo, position, date, time and type of violation and by enabling \textit{Authorities} to request them.
\\
\\
Safestreets requires the users to create an account to access its services, the functionalities unlocked after 
registration depend on the type of account created.
\\
If a user creates an account as \textit{Citizen}, he/she must provide information about ID card on order to prove 
that he/she is a real person. Furthermore, he must provide an email with which he will be uniquely identified 
and a password. Once the account has been activated, user can finally start to report parking violation. 
The users can also see a summary of the streets with the highest frequency of violations.
\\
\\
On the other hand, an officer will create an account as \textit{Authority} and he will need to provide his name, surname, 
work's Matricola, a password and as for \textit{Citizen}, will be uniquely identified by an email. Once the Matricola 
has been verified and the account has been activated, the officer can retrieve the potential parking violations 
sent by \textit{Citizen} that have not been taken into account yet by other officers, analyze them and, if it is the 
right case, generate traffic tickets. \textit{Authorities} can also see a summary of the vehicles' license plate that 
commit the most violations.
\\
\\
From this brief description of the functionalities we may extract the following goals for SafeStreets:
\begin{itemize}
    \item \textbf{[G1]}: allow users to be identified as a \textit{Citizen} or as \textit{Authority};
    \item \textbf{[G2]}: allow \textit{Citizens} to report parking violations;
    \item \textbf{[G3]}: \textit{Citizen} has to be able to input information about the violation that he has reported;
    \item \textbf{[G4]}: must provide a visualization of the streets with high frequency of violations and vehicles' license plate that commit the most violations;
    \item \textbf{[G5]}: \textit{Authority} can retrieve traffic violantions' data inserted by \textit{Citizens}
\end{itemize}

\subsection{Scope}
Here we will describe all the relevant phenomena that may occur. 

\subsubsection{World Phenomena}
Those are the events that may occur in the real word and are not affected by the Machine.
\\We identify:
\begin{itemize}
    \item \textit{Citizen} sees a parking violation and wants to report it;
    \item \textit{Authorities} want to know about some violations that have been occurred;
    \item A \textit{parking violation} occurs; 
\end{itemize} 

\subsubsection{Shared Phenomena}
Shared phenomena are the events based on the link beetween World Phenomena and Machine Phenomena.
We can distinguish them in two types:
\\
Controlled by the world observed by the machine:
\begin{itemize}
    \item A Citizen take a photo of a violation;
    \item Users can enter data for registration/login;
    \item Users can request data;
\end{itemize}
Controlled by the machine observed by the world:
\begin{itemize}
    \item track position of the violation;
    \item mark areas with an high rate of violations;
    \item System can fullfill data requests;
\end{itemize}

\subsubsection{Machine Phenomena}
The Machine Phenomena are the events that occur inside the machine and are not affected by the real world.
\\We identify:
\begin{itemize}
    \item storing permanently collected data;
    \item encryption of sensitive data;
    \item retrieving data for a request; 
\end{itemize} 

\subsection{Definitions, acronyms, abbreviations}

\subsubsection{Definitions}
\begin{itemize}
    \item user: can be either citizen or authorities
\end{itemize}
traffic violation: generic violation that can occur in a street
parking violation: a violation caused by a bad parking
violation: general violation, identity both traffic or parking violation
unsafe areas: areas with an high rate of violations

\subsubsection{Acronyms}
Table with all acronyms used in document.
\begin{center}
\begin{tabular}{ | l | l |}
    \hline
    ACRONYM & COMPLETE NAME \\
    \hline
    RASD & Requirements Analysis and Specification Document \\
    \hline
    GPS & global positioning systems \\
    \hline
    ID & Identity document \\
    \hline
    S2B & Software to be \\
    \hline
    GDPR & General Data Protection Regulation \\
    \hline 
\end{tabular}
\end{center}

\subsubsection{Abbreviations}

\subsection{Revision History}

\subsection{Reference documents}

\subsection{Document Structure}
\begin{itemize}
    \item \textbf{Chapter 2:}
    \item \textbf{Chapter 3:}
    \item \textbf{Chapter 4:}  
\end{itemize}
%%%%%%%%%% !CHAPTER 1 %%%%%%%%%%

%%%%%%%%%% CHAPTER 2 %%%%%%%%%%
\section{Overall Description}

\subsection{Product perspective}
This section aims to explain in more detail the World, Machine and Shared Phenomena described in the 
previous Chapter. 
\subsubsection{World Phenomena}
\begin{itemize}
    \item \textbf{Citizen sees a parking violation and wants to report it:}
    While the Citizen is quietly walking, he sees a parking violations like a double parking or a car
    parked in the middle of bike lane so he logs in, takes a picture, inserts the type of violations
    and sends it to the \textit{System}.  
    \item \textbf{Authorities want to know abount some violations that have been occurred:}
    An Authority logs in and retrieves all the violations that have been send by the Citizen 
    \item \textbf{Parking violations occurrs:}
    Someone in the city decides to not follow parking rules and does not park his car in a proper way   
\end{itemize}
\subsubsection{Machine Phenomena}
\begin{itemize}
    
\item \textbf{Storing permanently collected data:}
The system needs to store, in a secure way, all the data submitted.
In order to achive this purpose and guarantee the best service
the system needs to use a DBMS.

\item \textbf{Encryption of sensitive data:}
Personal user's data and all the data relative to the violations
that can only be seen by authorities need to be encrypted in order
to proctect it from non-allowed third parties. 

\item \textbf{Retrieving data for a request:}
System have to fullfill the data request from the users. Data 
requests can be of two types, a Citizen request who want to see
statistics of a certain city area or data request by Authorities
who want to receive the violation reports collected by SafeStreet
\end{itemize}

\subsubsection{Shared Phenomena}
Controlled by the World observed by the Machine
\begin{itemize}
    \item \textbf{A Citizen take a photo of a violation:}
    Situation in which a Citizen spots a generic violation and wants
    to report it through the application. Using the phone camera he
    can take the photo of the violation.

    \item \textbf{User can enter data for registration/login:}
    A user decide to use the application and provides his personal 
    in order to register if it's the first time he use the app, or to
    identify himself.

    \item \textbf{Users can request data:}
    In this phenomena we make a distinction between Citizen and Authorities.
    A Citizen or may want to see violation statistics 
    of a certain area, or in the case of Authorities they can, request
    violation statistics and additional informations 
\end{itemize}

Controlled by the Machine observed by the World
\begin{itemize}
    \item\textbf{Track position of the violation}
    The \textit{System} can retrieve the position where the violation occurred by fetching it from GPS service
    \item\textbf{Mark areas with an high rate of violations}
    Once lots of violations occured, the \textit{System} mines the information that it has in order to highlight the areas with the highest frequency of violations
    \item\textbf{System can fullfill data requests}
    After processing a request, the \textit{System} will show to the user the result of the DBMS query in a proper way

\end{itemize}

\subsection{Product functions}
In this section are explained the functions associated to User.
\begin{itemize}
 \item \textbf{Citizen functions:}
 
 \textbf{Report a violation}
 When a Citizen sees a parking violation occurs, he takes a picture of the vehicle paying attenction to focus on the 
 license plate, inputs the type of the violation and sends it. The System will provide to add the position retrieving 
 from GPS, to add the right time and date and to add the license plate obtained through the algorithm. 

 \textbf{Retrieve statistics about unsafe areas}
Safestreets enables Citizen to visualize statistics about unsafe areas. SafeStreets mines the informations it has and let 
the Citizen retrieves the result through a clear interface containing significant plots, tables and charts. 

 \item \textbf{Authority functions:}
 
 \textbf{Retrieve statistics about unsafe areas}
 Safestreets enables Authority to visualize statistics about unsafe areas. SafeStreets mines the informations it has and let 
 the Authority retrieves the result through a clear interface containing significant plots, tables and charts. For istance 
 Authority can understand which are the vehicles that commit the most violations or who are the most egregious offenders.
  
 \textbf{Request violations data for traffic tickets}
SafeStreets enables Authority to retrieve all the parking violations sent by Citizens. For each parking violation Authority can
 accepts it or declines it. In the first case he can generates traffic ticket, in the second case he discard the informations about
 the parking violations. In both cases SafeStreets records response in order to build statistics.

\end{itemize}

\subsection{User characteristics}
Below we describe the convention used to identify the user of the application and the function that those 
users are allow to perform.
\begin{itemize}
    \item \textbf{Guest:} A user that have donwload the application but is not 
    registered yet. This type of user is not allowed to access 
    the application functionalities.
    \item \textbf{Citizen:} is a generic user app not related to authorities, a 
    common Citizen that want to use the application. After the
    registration process and the validation of the ID card provided
    He can log in the application and use the functionalities:
    - report a violation
    - request informations about the statistics of a certain area.
    \item \textbf{Authorities:} This user is associated to the local municipal
    police district, any traffic warden, once registered with 
    is matricola number and logged in have access to those 
    functionalities:
    - request informations about the statistics of a certain area.
    - request all the violations reported from Citizens. 
    \item \textbf{User:} can be both a Citizen or Authority type, in this document
    this name is used when it's not necessary make a distinction 
    between the two.
\end{itemize}

\subsection{Assumption and Dependencies Constraints}
\subsubsection{Assumption}
The following list present all the domain assumption made.
\begin{itemize}
    \item \textbf{[D1]}: Users can't make more than one account.
    \item \textbf{[D2]}: The Citizen assumes all responsibility for misrepresentation.
    \item \textbf{[D3]}: Citizens who use the application are evenly distributed in any city area.
    \item \textbf{[D4]}: The ID card present by the Citizen during the registration is valid.
    \item \textbf{[D5]}: All the violantions reported are valid.
    \item \textbf{[D6]}: The S2B allows to take photos from the application.
\end{itemize}

\subsubsection{Dependencies}
This list below represent all the dependencies that S2B need in order to work properly.
\begin{itemize}
    \item A internet connection.
    \item A Photocamera (with a minimum precision of 3Mp ?(non so se va messo, algoritmo può non riconoscere low quality foto))
    \item A GPS.
    \item A Trusted external Storage for violantions data   
\end{itemize}

\subsubsection{Constraints}
\begin{itemize}
    \item The S2B must guarantee the European data protection GDPR for user's sensitive data.
    \item The S2B will be used only in Italy due to personal data type like (fiscal code and police matricola).
\end{itemize}
%%%%%%%%%% !CHAPTER 2 %%%%%%%%%%

%%%%%%%%%% CHAPTER 3 %%%%%%%%%%
\section{Specific Requirements}

\subsection{External Interface Requirements}
\subsubsection{User Interfaces}
\subsubsection{Hardware Interfaces}
The System does not offer any Hardware Interfaces
\subsubsection{Software Interfaces}
As mobile applications, the main software interfaces are:
\begin{itemize}
    \item iOs
    \item Android
\end{itemize}
\subsubsection {Communication Interfaces}
\textbf{HTTPS protocol}: to safely communicate through the internet
\subsection{Functional Requirements}

\subsection{Performance Requirements}

\subsection{Design Constraints}
\subsubsection{Standards compliance}
\subsubsection{Hardware limitations}

\subsection{Software System Attributes}
\subsubsection{Reliability}
\subsubsection{Availability}
\subsubsection{Security}
\subsubsection{Maintainability}
\subsubsection{Portability}

\section{Formal Analysis with Alloy}

\section{Efforts}




%%%%%%%%%% !CHAPTER 3 %%%%%%%%%%
    
\end{document}