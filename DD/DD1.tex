\documentclass{article}

\usepackage[export]{adjustbox}
\usepackage{listings}
\usepackage{subcaption}
\usepackage{wrapfig}
\usepackage[dvipsnames]{xcolor}
\usepackage{alloy-style}
\usepackage{float}
\usepackage{graphicx}
\usepackage[utf8]{inputenc}
\usepackage[a4paper, top=4cm, bottom=4cm, left=4cm, right=4cm]{geometry}

\title{
    \textbf{\textit{SafeStreets}} \\
    \textbf{DD document}}

\date{Academic year: 2019 - 2020}
\author{
    Dario Miceli Pranio \\
    Pierriccardo Olivieri
}

\begin{document}
\pagenumbering{gobble}

\maketitle

%%%%%%%%%% LOGO POLIMI %%%%%%%%%%
\begin{figure}[h!]
    \centering
    \includegraphics[scale=0.5]{img/logo.png}
\end{figure}

\newpage
\pagenumbering{arabic}
\tableofcontents

\newpage
%%%%%%%%%% CHAPTER 1 %%%%%%%%%%
\section{Introduction}

\subsection{Purpose}
\textit{SafeStreets} is a service that aims to provide \textit{Users} with the possibility to notify \textit{Authorities} when traffic 
violations occur, and in particular parking violations. The application's goal is achieved by allowing \textit{Users} 
to share photo, position, date, time and type of violation and by enabling \textit{Authorities} to request them.
\\
\\
\textit{SafeStreets} requires the \textit{Users} to create an account to access its services, the functionalities unlocked after 
registration depend on the type of account created.
\\
If a \textit{User} creates an account as \textit{Citizen}, he/she must provide name, surname and a fiscal code in order to prove 
that he/she is a real person. Furthermore, he must provide an email with which he will be uniquely identified 
and a password. Once the account has been activated, \textit{User} can finally start to report parking violations and can also see 
statistics of the streets or the areas with the highest frequency of violations.
\\
\\
On the other hand, an officer will create an account as \textit{Authority} and he will need to provide his name, surname, 
work's Matricola, a password and as for \textit{Citizen}, will be uniquely identified by an email. Once the Matricola 
has been verified and the account has been activated, the officer can retrieve the potential parking violations 
sent by \textit{Citizen} that have not been taken into account yet by other officers, analyze them and, if it is the 
right case, generates traffic tickets. \textit{Authorities}, can see the same statistics of the \textit{Citizen} and can also see
statistics about vehicles' license plate that commit the most violations.


\subsection{Scope}

\subsection{Definitions, acronyms, abbreviations}

\subsubsection{Definitions}
\begin{itemize}
    \item \textit{Users}: can be either \textit{Citizen} or \textit{Authority}
    \item \textit{traffic violation}: generic violation that can occur in a street
    \item \textit{parking violation}: a violation caused by a bad parking
    \item \textit{violation}: general violation, identity both traffic or parking violation
    \item \textit{unsafe areas}: areas with an high rate of violations
\end{itemize}

\subsubsection{Acronyms}
Table with all acronyms used in document.
\begin{center}
\begin{tabular}{ | l | l |}
    \hline
    ACRONYM & COMPLETE NAME \\
    \hline
    DD & Design Document \\
    \hline
    GPS & Global Positioning Systems \\
    \hline
    S2B & Software To Be \\
    \hline
    GDPR & General Data Protection Regulation \\
    \hline 
    FC & Fiscal Code \\
    \hline
    UC & Use Case \\
    \hline
\end{tabular}
\end{center}

\subsubsection{Abbreviations}
\begin{itemize}
    \item \textbf{Rn}: n-th Requirement 
\end{itemize}

\subsection{Revision History}

\subsection{Reference documents}
\begin{itemize}
    \item ISO/IEC/IEEE 29148: https://www.iso.org/standard/45171.html
    \item Specification Document: "SafeStreets Mandatory Project Assignement"
\end{itemize}

\subsection{Document Structure}
\begin{itemize}
\end{itemize}
%%%%%%%%%% !CHAPTER 1 %%%%%%%%%%

\clearpage
\end{document}